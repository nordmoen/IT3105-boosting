\subsection{Pen Digit Dataset}\label{pen digit dataset}
This dataset represents the classification of digits written by different writers.

To run, use these options with the classifier options wanted.

\begin{lstlisting}[label=lst:pen, caption=Pen digit dataset general options]
--global 0.2 42 --file pen-digits.txt --filter no.ntnu.ai.filter.PenDigitsFilter
\end{lstlisting}

\begin{landscape}
\begin{table}
\begin{tabular}{|c|c|c||c|c|c||c||p{5cm}|}
\hline
NBC \# & Training Error & Standard Deviation & DTC \# & Training Error
& Standard Deviation & Test Error & Classifier option \\ \hline
1 & 0.119 & N/A & 0 & N/A & N/A & 251/2199(11\%) & NBCGenerator 1 \\ \hline
0 & N/A & N/A & 1 & 0.0 & N/A & 252/2119(11\%) & DTCGenerator 1 \\ \hline
5 & 0.314 & 0.106 & 0 & N/A & N/A & 248/2199(11\%) & NBCGenerator 5 \\ \hline
10 & 0.329 & 0.081 & 0 & N/A & N/A & 239/2119(10\%) & NBCGenerator 10 \\ \hline
20 & 0.330 & 0.078 & 0 & N/A & N/A & 230/2119(10\%) & NBCGenerator 20 \\ \hline
0 & N/A & N/A & 5 & 0.0 & 0.0 & 252/2119(11\%) & DTCGenerator 5 \\ \hline
0 & N/A & N/A & 10 & 0.761 & 0.057 & 1126/2119(55\%) & DTCGenerator 10 1 \\ \hline
0 & N/A & N/A & 10 & 0.615 & 0.127 & 398/2119(18\%) & DTCGenerator 10 2 \\ \hline
0 & N/A & N/A & 10 & 0.0 & 0.0 & 252/2119(11\%) & DTCGenerator 10 \\ \hline
0 & N/A & N/A & 20 & 0.0 & 0.0 & 252/2119(11\%)& DTCGenerator 20 \\ \hline
5 & 0.289 & 0.086 & 5 & 0.664 & 0.063 & 172/2119(7\%) & DTCGenerator 5 2, 
\newline NBCGenerator 5 \\ \hline
10 & 0.351 & 0.049 & 10 & 0.612 & 0.0121 & 196/2119(8\%) & DTCGenerator 10 2, 
\newline NBCGenerator 10 \\ \hline
20 & 0.347 & 0.073 & 20 & 0.656 & 0.091 & 174/2119(7\%) & DTCGenerator 20 2, 
\newline NBCGenerator 20 \\ \hline
\hline
\end{tabular}
\label{tab:pen}
\caption[Pen-digits dataset boosting]{Table showing the results of our 
classifiers on the Pen-digits dataset}
\end{table}
\end{landscape}

This is one of the only datasets which we can see that boosting is actively
helping the solution get better. As we can see the bottom three rows does
somewhat better than each of the others individually. This can only come from
the effect of boosting as neither the decision tree nor the naive Bayes does
better individually than combined. However it does seem that the number of
classifiers has little to say, but the combination of decision tree and naive
Bayes is what makes this better. If we also look at only five, ten and twenty
NBCs we can see that the result does get marginally better, but it's not enough
to call it better for our booster.
