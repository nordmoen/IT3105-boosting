\section{Implementation}\label{implementation}
In this section we will talk a bit about our implementation. We will describe
how it manages to classify new instances, how we generate new classifiers and
how it can be extended without changing any implementation.

From our \adaboost{} implementations view all we generate are Hypothesis\footnote{
no.ntnu.ai.hypothesis.Hypothesis} which comes from Generators\footnote{
no.ntnu.ai.hypothesis.Generator}. These interfaces dictates what all new
classifiers must support. A Generator creates a new Hypothesis with the basis
in weights and the training set. An Hypothesis only needs to support setting
its weight and classifying a new instance\footnote{We have created an abstract
class for convenience, situated in no.ntnu.ai.classifiers.Classifier}.

From an Hypothesis point of view all data is of type DataElement\footnote{
no.ntnu.ai.data.DataElement} which contains a number of attributes specified when
it's created. It also contains a classification which it must be able to produce.
DataElement uses Java generics to support different types of attributes and 
classifications which is determined by the filter used. Each Generator must
support the same attributes and classifications as the DataElement used, but
this is only checked at runtime.

Our data flow starts with reading all lines in a file creating DataElements
with String values. Then we apply a filter which must implement the interface
Filter\footnote{no.ntnu.ai.filter.Filter}. After that is done we shuffle the
dataset using a specified random seed to be able to replicate our tests. We
then split the dataset into a training set and test set with a split specified
on the command-line. After that we start \adaboost{} and try to classify the
test set.

The meat of our implementation lies in the Generator class which dictates how
a specific classifier is created. This is the class which is defined on the
command-line and this is the class which can be configured from the command-line.

This class must be able to receive a list of options where it can act on each
option according to its own wishes which is done so we can configure each Generator,
this is among other things where we pass the depth parameter to the Decision Tree
Generator.

Each Generator class uses its own logic to create a classifier of the correct type
based on the given data. This means that \adaboost{} does not know nor care what
a Generator creates as long as it implements the Hypothesis interface. This also
means that our implementation is very general. When creating a new classifier type
one has to implement a new Generator and a new Hypothesis class, which is all. To
support new datasets one only has to implement a new filter to convert from String
to a specific type. This was done to decouple the datasets from the classifiers, but
it also means that we have to implement less code since each classifier shares the
filters.

The filters are tasked with both converting the String values read from a file to
the correct format, but it's also tasked with creating buckets for the classifiers.
With this we mean that the filter must split the values in each file into proper
buckets since our classifiers can't handle large ranges. This is also the reason
why we have a filter per dataset. Since we had a hard time creating an algorithm
for splitting the ranges in different datasets we chose to have it like this for ease
and to be able to control the buckets in more detail.
